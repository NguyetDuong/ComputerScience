\documentclass[11pt,letterpaper]{article}
\usepackage{amsmath}
\usepackage{amssymb}
\usepackage{circuitikz}
\usepackage{fullpage}

\title{CS 70, Spring 2015 --- Homework 2} % Put the correct homework number here.
\author{Nguyet Minh Duong, SID 24444722} % Put your name and student ID here.

\begin{document}

\maketitle

\section*{Problem 1} 
\begin{enumerate}
	\item[1a.] $\forall x P(x) \equiv \neg \exists x \neg P(x)$ 

	TRUE because if we follow De Morgan's Law, this happens:
	\begin{enumerate}
		\item[] $\neg \exists x \neg P(x) \equiv \forall x \neg \neg P(x)$
		\item[] $\forall x \neg \neg P(x) \equiv \forall x P(x)$
		\item[] Therefore, after simplifying: $\forall x P(x) \equiv \forall x P(x)$
	\end{enumerate}
	\item[1b.] $\forall x \exists y P(x,y) \equiv \forall y \exists x P(x,y)$ 
	
	FALSE, using the counterexample: $P(x,y)$ will be true if $y > 0$, and for all x, the other statement cannot be true since we already decided that $P(x,y)$ can ONLY be try for when $y > 0$. 
	\item[1c.] $P \Rightarrow Q \equiv \neg P \Rightarrow Q$
	
	FALSE, using counterexample: assume P is False and Q is False. 
	\begin{enumerate}
		\item[] Then $P \Rightarrow Q \equiv TRUE$
		\item[] However $\neg P \equiv TRUE$
		\item[] Therefore $\neg P \Rightarrow Q \equiv FALSE$
		\item[] Hence $TRUE \not\equiv FALSE$
	\end{enumerate}		
	\item[1d.] $(P \Rightarrow Q) \land (\neg P \Rightarrow \neg Q) \equiv P \Leftrightarrow Q$
	
	TRUE, from the following proof:
	\begin{enumerate}
		\item[] $A :=$ left side of the statement
		\item[] $B :=$ right side of the statement
		\item[] We assume $A \equiv TRUE$. 
		\item[] $A$ can only be true if $P \equiv TRUE, Q \equiv TRUE$ or $P \equiv FALSE, Q \equiv FALSE$
		\item[] Then $B$ is true proven by the following truth table:
		
		\begin{tabular}{ l l l l l }
			$P$ & $Q$ & $P \Rightarrow Q$ & $Q \Rightarrow P$ & $P \Leftrightarrow Q$ \\ \hline
			T & T & T & T & T\\
		    F & F & T & T & T\\
		\end{tabular}
		\item[] Similarly, $B \equiv TRUE$ when $P \equiv TRUE, Q \equiv TRUE$ or $P \equiv FALSE, Q \equiv FALSE$
	\end{enumerate}
\end{enumerate}

\newpage

\section*{Problem 2} 
\begin{enumerate}
	\item[2a.] $\forall n P(x)$ is true. 
	
	This can hold but not always due to what we have proven: $\forall k \in \mathbb{N},$ if $P(k)$ is true, $P(k+2)$ is true. We skipped the consecutive number right after $k$, which is denoted as $k+1$. Therefore what we have proven is that for every other number after $k$ is true not every number after $k$ is true.
	\item[2b.] If $P(0)$ is true then $\forall n P(n+2)$ is true.
	
	This can hold but not always because in this statement, we set our $k = 0$, therefore only the following would be true under what we've proven: $k = 0 + 2 = 2, k = 2 + 2 = 4, k = 4 + 2...$. As we can see, all of these numbers are even. But when we say $\forall n$, this means any $n + 2$ is suppose to be true. A counterexample is when $n = 2n + 1$. As a result $2n + 1 + 2 = 2(n + 1) + 1$ which is odd. Therefore it does not work when $n$ is odd.
	\item[2c.] If $P(0)$ is true then $\forall n P(2n)$ is true. 
	
	This always holds. The statement set $k = 0$. As exampled above, all the numbers that we have proven is true are even numbers. $k = 2n$ is an even number. Therefore this statement is true because $\forall n, 2n$ is even by definition.
	\item[2d.] $\forall n P(n)$ is false.
	
	This can hold but not always. The statement is similar to 2a in the way that we have proven that we stated every other number is true. Therefore the numbers in between is false so half of it is true and half of it is true. 
	\item[2e.]If $P(0)$ and $P(1)$ are true then $\forall n P(n)$ is true.
	
	This always holds. Because now we assume $k$ can be two things $k = 1$ and $k = 0$. We already stated in what numbers $k = 0$ proves to be true, and it is all even numbers when $k = 0$ in 2b. Now we notice that any $k + 2$ when starting at 1 will be odd because of the definition of an odd number. Notice the pattern: $k = 1, k = 1 + 2 = 3, k = 3 + 2 = 5...$.
	\item[2f.] $(\forall n \leq 10$ $P(n)$ is true$)$ and $(\forall n > 20$ $P(n)$ is false$)$ 
	
	This never holds because allowing n be 1 or 2 and making the statement $P(1)$ and $P(2)$ is enough to set the foundation such that all of the natural numbers following will be true because 1 covers all the odd numbers and 2 covers all the even numbers.
	\item[2g.] $[\forall n$ $prime \Rightarrow P(n)] \Rightarrow [\forall n \geq 11$ $P(n)]$
	
	This always holds because as explained in 2e, number 2 will cover all of the even numbers. Similarly, the consecutive number, 3, will cover all of the odd numbers. Therefore any number after 3 will be true. 
\end{enumerate}

\newpage

\section*{Problem 3}
\begin{enumerate}
	\item[3a.] Prove that $3+11+19+...+(8n-5) = 4n^2-n$ for all integers $n \geq 1$.
	\begin{enumerate}
		\item[] We will proceed to prove this by induction.
		\item[] Base Case: $n = 1$, so $(8(1) - 5) = 3$. $4(1^2) - 1 = 3$
		\item[] Hypothesis: Assume for $n = k \geq 1$, $3+11+19+... +(8k-5) = 4k^2-k$
		\item[] Step: we want: $3+11+19+...+ (8k-5) + (8(k+1) -5) = 4(k+1)^2 - (k+1)$
		\begin{enumerate}
			\item Add $(8(k+1) -5)$ to both sides: 
			\\ $3+11+19+...+ (8k-5) + (8(k+1) -5) = 4k^2 - k + 8(k+1) - 5$
			\item Manipulate the right hand side:
			\\$LHS = 4k^2 - k + 8k + 8 - 5$
			\\$LHS = 4k^2 - k - 1 + 8k + 8 - 4$
			\\$LHS = 4k^2 + 8k + 4 - (k + 1)$
			\\$LHS = 4(k^2 + 2k + 1) - (k + 1)$
			\\$LHS = 4(k + 1)^2 - (k + 1)$
			\item Hence, what we originally wanted to prove. QED
		\end{enumerate}
	\end{enumerate}
	\item[3b.] Prove that $1^3 + 2^3 + 3^3 + ... + n^3 = (1 + 2 + 3 + ... + n)^2$ for all integers $n\geq 1$
	\begin{enumerate}
		\item[] We will proceed to prove this by induction.
		\item[] Base Case: $n = 1$, $1^3 = 1 = 1^2 = 1$
		\item[] Hypothesis: Assume for $n = k \geq 1$, $1^3 + 2^3 + 3^3 + ... + k^3 = (1 + 2 + 3 + ... + k)^2$
		\item[] Step: \begin{enumerate}
			\item Note that $(1 + 2 + 3 + ... + k)^2 = (\sum\limits_{i=1}^k i)^2 = (\frac{k(k+1)}{2})^2$
			\item Therefore what we want to prove is $1^3 + 2^3 + 3^3 + ... + k^3 + (k+1)^3 = (\frac{(k+1)(k+2)}{2})^2$
			\item Using original equation, we add $(k+1)^3$ to both sides:
			\\ $1^3 + 2^3 + 3^3 + ... + k^3 + (k+1)^3 = (\frac{k(k+1)}{2})^2 + (k+1)^3 $
			\item Now simplify the right hand side:
			\\ $LHS = (\frac{(k(k+1))^2}{4}) + \frac{4(k+1)^3}{4}$
			\\ $LHS = (\frac{k^2(k+1)^2 + 4(k+1)^3}{4})$
			\\ $LHS = (\frac{(k+1)^2(k^2 + 4(k+1)}{4})$
			\\ $LHS = (\frac{(k+1)^2(k^2 + 4k + 4}{4})$
			\\ $LHS = (\frac{(k+1)^2(k+2)^2}{4}))$
			\\ $LHS = (\frac{(k+1)(k+2)}{2})^2$
			\item Hence we've proven what we originally wanted to prove. QED
		\end{enumerate}
	\end{enumerate}
\end{enumerate}

\newpage

\section*{Problem 4}
Let $x \in \mathbb{R}$ be such that $a_1 = x + \frac{1}{x} \in \mathbb{Q}.$ Using strong induction, show that for each integer $n \geq 1$, $a_n = x^n + \frac{1}{x^n} \in \mathbb{Q}.$
\begin{enumerate}
	\item[] We will prove this by strong induction.
	\item[] Base Cases: 
	\\$n = 1$, $a_1 = x + \frac{1}{x} = \frac{x^2 + 1}{x}$
	\\ $n = 2$, $a_1 = x^2 + \frac{1}{x^2} = \frac{x^4 + 1}{x^2}$
	\begin{enumerate}
		\item Notice how $a_1$ and $a_2$ looks really similar. We will attempt to represent it in a way that uses the previous term(s) because this is \textit{strong} induction.
		\item We will now attempt to manipulate $a_1$ such that it will look like $a_2$:
		\begin{enumerate}
			\item[] Notice the $x^4$ at the top, and $x^2$, so we might want to square $a_1$
			\\$a_1^2 = \frac{(x^2 + 1)^2}{x^2} = \frac{x^4 + 2x^2 + 1}{x^2} = \frac{x^4 +1}{x^2} + \frac{2x^2}{x^2} = \frac{x^4 +1}{x^2} + 2 = a_2 + 2$
			\\ Rearranging for $a_2 = a_1^2 - 2$
			\\ We can now say that $a_2 \in \mathbb{Q}$ because $\mathbb{Q}$ is closed under multiplication and addition.
		\end{enumerate}
	\end{enumerate}
	$n = 3$, $a_3 = x^3 + \frac{1}{x^3} = \frac{x^6 + 1}{x^3}$
	\begin{enumerate}
		\item Again, notice how $a_3$ looks very similar to both $a_2$ and $a_1$. Using this, we will attempt to represent it in a way that uses the previous term(s).
		\item We will attempt to manipulate $a_1$ to look like $a_3$:
		\begin{enumerate}
			\item[] Notice $x^6$ at the top and $x^3$ at the bottom. We can get this by $a_1 * a_2$
			\\ $a_1 * a_2 = \frac{x^2 + 1}{x} * \frac{x^4 + 1}{x^2} = \frac{(x^2 + 1)(x^4 + 1)}{x^3} = \frac{x^6 + x^2+ x^4 + 1}{x^3} = \frac{x^6 + 1}{x^3} + \frac{x^2 + x^4}{x^3} = \frac{x^6 + 1}{x^3} + \frac{1+ x^2}{x} = a_3 + a_1 $
			\\ Rearranging for $a_3 = a_1 * a_2 - a_1$
			\\ We can now say that $a_3 \in \mathbb{Q}$ because $\mathbb{Q}$ is closed under multiplication and addition.
		\end{enumerate}
	\end{enumerate}
	From this we can realize that we need to use some form of strong induction/proof stating that it can use the previous terms, which we have proven above. 
	\item[] Hypothesis: assume $n = k$, $a_k = x^k + \frac{1}{x^k} = \frac{x^{2k} + 1}{x^k}$ and that for all $n = j$ such that for all $1 \leq j \leq k$, $a_j = x^j + \frac{1}{x^j} = \frac{x^{2j} + 1}{x^j} \in \mathbb{Q}$
	\item[] Step: 
	\begin{enumerate}
		\item We want: $a_{k+1} = x^{k+1} + \frac{1}{x^{k+1}} = \frac{x^{2(k+1)} + 1}{x^{k+1}}$
		\item We need to manipulate $a_k$ so that it looks like what we want. Notice the top is x gets added one to the exponent and the bottom one adds one exponent. So we attemp the following, using the previous term $a_1$:
		\\ $a_k * a_1 = \frac{x^{2k} + 1}{x^k} * \frac{x^2 + 1}{x} = \frac{(x^{2k} + 1)(x^2 + 1)}{x^{k+1}} = \frac{x^{2k + 2} + x^2 + x^{2k} + 1}{x^{k+1}} = \frac{x^{2k + 2} + 1}{x^{k+1}} + \frac{x^2 + x^{2k}}{x^{k+1}} = \frac{x^{2k + 2} + 1}{x^{k+1}} + \frac{x^{2k-2} + 1}{x^{k-1}}$
		\\ Now we have $a_k * a_1 = a_{k+1} + \frac{x^{2k-2} + 1}{x^{k-1}}$
		\item Notice how $\frac{x^{2k-2} + 1}{x^{k-1}}$ follows a pattern... when we input $k = 3$, we get $a_2$, and when we input $k = 2$, we get $a_1$. Therefore this exists and it is known as $j$. 
		\item We can now rewrite the formula as: $a_k * a_1 = a_{k+1} + a_{k-1}$
		\item And therefore $a_{k+1} = a_k * a_1 - a_{k-1}$ which $\in \mathbb{Q}$ because $\mathbb{Q}$ is closed under addition and multiplication.
		\item Hence we've proven that  $a_n = x^n + \frac{1}{x^n} \in \mathbb{Q}$, $\forall n \geq 1$. QED
	\end{enumerate}	 
\end{enumerate}

\newpage

\section*{Problem 5}
Let $a_0 = 1$ and $a_n = 2a_{n-1} + 7$. Prove that there is a constant $C > 0$, which does not depend on $n$, such that $a_n \leq C * 2^n$ for all $n \in \mathbb{N}$.
\begin{enumerate}
	\item Base Case: $a_0 = 1$ and $a_1 = 2(a_0) + 7 = 9$
	\item Hypothesis: (strengthen) there exists $n = k$ such that $a_k \leq C * 2^k - 7$, choosing 7 because it matched the equation. 
	\item Step: 
	\begin{enumerate}
		\item We want $a_{k+1} \leq C * 2^{k+1} - 7$
		\item Starting with $a_k \leq C * 2^k - 7$, we change the left side with $a_{k+1}$ and try to manipulate it such that the right side looks like what we want.
		\item $C * 2^{k} - 7$, notice that for $a_{k+1}$, the $2^{k+1}$, we can get this if we multipled the whole RHS by 2: 
		\\ $2a_k \leq 2(C * 2^k - 7) = C * 2^{k+1} - 14 $
		\item Notice that we are 7 away from the equation we want, so we add 7 to both sides:
		\\ $2a_k + 7 \leq C * 2^{k+1} - 7 $
		\item Now, we proved that $a_{k+1}$ exists through the fact that we can manipulate $a_k$ to become this. 
		\item The reason this strengthening works is because $C * 2^n - 7 < C * 2^n$. Therefore if it is less that $C * 2^n - 7$, then it is definitely less than $C * 2^n$. QED
	\end{enumerate}	 
\end{enumerate}

\newpage

\section*{Problem 6}
\begin{enumerate}
	\item[6a.] F because the base case is incorrect. We did not plug 1 into the equation to prove that the equation works in the first place.
	\item[6b.] F because the person is using strong induction however the person did not include additional base cases to prove that it the base cases use each other. 
\end{enumerate}

\newpage

\section*{Problem 7}
\begin{enumerate}
	\item[] Base Cases: $n = 0 \rightarrow x^0 = 1$
	\\ $n = 1 \rightarrow x^1 = x * x^0 = x^1$
	\\ $n = 2 \rightarrow x^2 = (x^{\frac{2}{2}})^2 = x^2$
	\\ $n = 10 \rightarrow x^{10} = (x^{\frac{10}{2}})^2 = (x^5)^2$
	\\ $n = 5 \rightarrow x^5 = x*x^4 $
	\\ $n = 4 \rightarrow x^4 = (x^{\frac{4}{2}})^2 = (x^2)^2$
	\item[] Hypothesis: there exists a $k$ such that $x^k = Power(x,k)$ and there exists $j = 0, 1, 2, ..., k$ such that $x^j = Power(x,j)$.
	\item[] Steps:
	\begin{enumerate}
		\item Want $Power(x,k+1) = x^{k+1}$
		\item Case 1 (k is odd): $Power(x, k+1) = x * x^k$.
		\\ This works because of our hypothesis!
		\item Case 2 (k is even): $Power(x, k+1) = (x^{\frac{k}{2}})$.
		\\ This works because of our hypthosis!
		\\ Case 3 $(k + 1 = 0)$: this is not possible because that would make $k = -1$ which is not possible.
		\item Therefore we covered all the cases and proven they work from the hypothesis. QED
	\end{enumerate}
\end{enumerate}

\newpage

\section*{Problem 8}
Doing this problem out by hand for the first 4 plates, we see a pattern that will be represented in the base cases.
\begin{enumerate}
	\item[] Base Cases: 0 plates = 0 moves
	\\ 1 plate = 1 move
	\\ 2 plates = 3 moves
	\\ 3 plates = 7 moves
	\\ 4 plates = 15 moves
	\item[] Hypothesis: There exists a $k$ plates such that it will take $2^k - 1$ moves.
	\item[] Steps:
	\begin{enumerate}
		\item We want to prove: for $k+1$ plates, it will take $2^{k+1} - 1$ moves.
		\item So we start off with moving the first k plates on needle A off the biggest plate (number $k+1$ plate). This takes $2^k - 1$ as we said above so that it will end up on all on one needle. For this instance, make that needle C.
		\item Now we move all the $k+1$ plate to needle B. This takes an additional move.
		\item Then we have to remove the rest of the plates in the same manner previously such that it will stack upon the $k+1$ plate in order. This takes $2^k - 1$ as we have determined earlier. 
		\item Therefore it is: $2(2^k - 1) + 1 = 2^{k+1} - 1$ QED
	\end{enumerate}
\end{enumerate}
Hence, plugging in 64, seeing that this is the pattern. We will get around 584 billion years because the end of the earth.
\end{document}
